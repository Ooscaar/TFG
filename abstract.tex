\section*{Abstract}
Containers are an operating system virtualization technology used to provide processes isolate environments. They provide a lightweight solution where a single Linux kernel is shared between the host and the containers.

Using this technology several project have emerged over the years which offer different use cases.
One of these project is ''LXD``, which offers a way of running unmodified Linux distributions inside containers.

This thesis intends to provide a framework on top of ``LXD'' in order to improve some of it's
functionalities and provide new user cases.

We have developed two command line tools and one minimal web application which integrate all the improvements on top of ``LXD''.
\addcontentsline{toc}{section}{Abstract}


\newpage
\section*{Resum}
Els contenidors són una tecnologia utilitzada per el sistema operatiu que ens permet oferir alsprocessos un entorn aïllat. Ofereixen una tecnologia molt lleugera on només es comparteix un únix kernel entre el propi host i els contenidors.

A partir d'aquesta tecnologia diversos projectes han surgit al llarg dels anys. Un d'aquests projectes és ``LXD'', que permet utilizar distribucions Linux dintre de contenidors.

Aquesta tesis té l'objectiu d'oferir un sistema que utilitzant ``LXD'' permeti extendre i millorar les propies funcionalitats.

S'han desenvolupat dues aplicaciones de comandes i una aplicació web que utilitzant ``LXD'' han permès millorar les funcionalitats de ``LXD''.

\addcontentsline{toc}{section}{Resum}


\newpage
\section*{Resumen}
Los contenedores son una tecnología empleada por el sistema operativo que permite ofrecer
a los procesos un entorno aislado. Ofrecen una solución muy ligera donde
únicamente se comparte un kernel entre el propio host y los contenedores.

A partir de esta tecnología, diversos proyectos han surgido durante los últimos años que
emplean dicha tecnología. Uno de estos proyecto es ``LXD'', que permite correr distribuciones de Linux no modificadas dentro de contenedores.

Esta tesis tiene el objectivo de ofrecer un sistema que usando ``LXD'' permite extender y mejoras sus funcionalidades.

Para eso, hemos desarollado dos comandos y una aplicación web que usando ``LXD'' nos han permitido mejorar dichas funcionalidades.

\addcontentsline{toc}{section}{Resumen}
