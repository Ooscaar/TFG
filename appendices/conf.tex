%%%%%%%%%%%%%%%%%%%%%
%% APPENDIX-LXCE-CONF %%%%
%%%%%%%%%%%%%%%%%%%%%
\newpage\newpage\section{Configuration files}

\begin{listing}[H]
\begin{minted}[bgcolor=background]{text}
/etc/lxce 
|--- container.conf.d 			
|   |--- default 			
|   |   '--- voiceless-blue
|   '--- derecho 			
|       '--- relieved-beige
|--- container_default.conf 		
|--- lxce.conf 			
|--- remmina 		
|   |--- default 
|   |   '--- oscar-vm.default.voiceless-blue.remmina
|   '--- derecho 
|       '--- oscar-vm.derecho.relieved-beige.remmina
'--- ssh 	
    |--- default 
    |   '--- voiceless-blue.conf
    '--- derecho
        '--- relieved-beige.conf
\end{minted}
\caption{lxce directory structure}
\label{listings: lxce directory structure /etc/lxce}
\end{listing}

In this way we are able to manage the container configurations from our command line and update/delete files based on the state of the command.

Where the configurations files content is the following:
\begin{itemize}
%%%%%%%%%%
%% ITEM %%
%%%%%%%%%%
\newpage
\item{\textbf{container-default.conf}

This file acts as a template for every container to be created.
\begin{listing}[H]
\begin{minted}[bgcolor=background]{json}
{
  "name": "",                           
  "alias": "",                          
  "user": "",
  "id_domain": 0,
  "id_container": 0,

  "domain": "default",                 
  "base": "ubuntu:20.04",             
  "userData": "/datasdd",            

  "proxies": [                      
    {
      "name": "ssh",
      "type": "tcp",
      "listen": "0.0.0.0",
      "port": 22
    },
    {
      "name": "test",
      "type": "tcp",
      "listen": "0.0.0.0",
      "port": 3000
    }
  ],
}
\end{minted}
\caption{/etc/lxce/container-default.conf}
\label{listings: /etc/lxce/container-default.conf}
\end{listing}
\TODO{Think the convention for the name of the figures and the descriptions of the listings}
}
%%%%%%%%%%
%% ITEM %%
%%%%%%%%%%
\newpage
\item{\textbf{lxce.conf}

This file specifies different parameters of the host where the command is installed, such as:
\begin{itemize}
	\item{SSH IP}
	\item{Hostname}
	\item{Local VNC server configuration}
	\item{Seed used for generating passwords}
	\item{List of container domains currently in the host}
	\item{List of locations available for the shared containers folders location}
\end{itemize}
\begin{listing}[H]
\begin{minted}[bgcolor=background]{json}
{
  "hypervisor": {
    "SSH_hostname": "localhost",
    "SSH_suffix": "oscar-vm",
    "VNC_server": "localhost",
    "VNC_port": 5901
  },
  "seed": "4b5a003f0e1715df",
  "domains": [
    {
      "id": 0,
      "name": "default"
    },
    {
      "id": 1,
      "name": "derecho"
    }
  ],
  "locations": [
    "/datasdd"
  ]
}
\end{minted}
\caption{lxce.conf}
\label{listings: /etc/lxce/lxce.conf}
\TODO{explain all the parameters}
\end{listing}
}
%%%%%%%%%%
%% ITEM %%
%%%%%%%%%%
\newpage
\item{\textbf{container configuration file}

This files list the configured parameters for each container and the ids that uniquelly identifies it
\begin{listing}[H]
\begin{minted}[bgcolor=background]{json}
{
  "name": "voiceless-blue",
  "alias": "",
  "user": "ubuntu",
  "id_domain": 0,
  "id_container": 0,
  "domain": "default",
  "base": "ubuntu:20.04",
  "userData": "/datasdd",
  "proxies": [
    {
      "name": "ssh",
      "type": "tcp",
      "listen": "0.0.0.0",
      "port": 22
    },
    {
      "name": "test",
      "type": "tcp",
      "listen": "0.0.0.0",
      "port": 3000
    }
  ],
}
\end{minted}
\caption{container configuration file}
\label{listings: /etc/lxce/container.conf.d/default/voiceless-blue}
\end{listing}
}
%%%%%%%%%%
%% ITEM %%
%%%%%%%%%%
\clearpage
\item{\textbf{VNC configuration}
\begin{listing}[H]
\begin{minted}[bgcolor=background]{text}
[remmina]
ssh_tunnel_privatekey=
name=oscar-vm.default.voiceless-blue          # Container name
ssh_tunnel_passphrase=
password=.                                    # For saving VNC password
...
server=localhost:5901                         # VNC server
disablepasswordstoring=0
ssh_tunnel_username=ubuntu
disableclipboard=0
window_maximize=1
ssh_tunnel_password=.                         # For saving ssh password
enable-autostart=0
proxy=
ssh_tunnel_server=localhost:10000             # Container SSH Port
ssh_tunnel_auth=0
group=oscar-vm.upc.edu                        
...
protocol=VNC
username=ubuntu                               # VNC username
showcursor=0
colordepth=32

\end{minted}
\caption{REMMINA configuration file}
\label{listing: /etc/lxce/remmina/default/oscar-vm.default.voiceless-blue.remmina}
\end{listing}
}
%%%%%%%%%%
%% ITEM %%
%%%%%%%%%%
\item{\textbf{SSH configuration}
\begin{listing}[H]
\begin{minted}[bgcolor=background]{text}
Host oscar-vm.default.voiceless-blue
   Hostname localhost
   User ubuntu
   Port 10000
   TCPKeepAlive yes
   ServerAliveInterval 300
\end{minted}
\caption{ssh configuration file}
\label{listings: /etc/lxce/ssh/default/voiceless-blue.conf}
\end{listing}
}
\end{itemize}
