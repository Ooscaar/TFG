%%%%%%%%%%%%%%%%%%%%%
%% APPENDIX-LXCE %%%%
%%%%%%%%%%%%%%%%%%%%%
\section{lxce}\label{annex:lxce}
For the commands that are available for our command, we have the following structure:
\begin{minted}[bgcolor=background]{text}
Usage: lxce [command] <options> <flags>

Commands:
  lxce alias       Manage containers aliases
  lxce completion  Output completions scripts
  lxce delete      Delete containers and configurations/folders related
  lxce init        Initialize lxce command
  lxce launch      Launch containers
  lxce list        List containers properties
  lxce pass        Compute password from containers
  lxce proxy       Delete and restart proxies 
  lxce rebase      Relaunch container with new base specified
  lxce show        Show containers configurations files
  lxce start       Start containers
  lxce stop        Stop containers
  lxce uninstall   Remove all configurations from the lxce command

Flags
      --version  Show version number        
  -h, --help     Show help                 
  -v, --verbose
\end{minted}

%%%%%%%%%%%%%%%%%%%%%%%%
%%% LIST OF COMMANDS %%%
\newpage
\textbf{lxce alias}
\begin{listing}[H]
\begin{minted}[bgcolor=background]{text}
Usage: lxce alias [command] <options> <flags>

Commands:
  lxce alias set    set container alias
  lxce alias unset  unset container alias
  lxce alias check  check container alias

Flags
      --version  Show version number                                   
  -h, --help     Show help                                             
  -v, --verbose
\end{minted}
\caption{lxce alias}
\label{listings: lxce alias}
\end{listing}

\textbf{lxce alias set}
\begin{listing}[H]
\begin{minted}[bgcolor=background]{text}
Usage: lxce alias set [options] <flags>

Options
  -d, --domain  container domain                             
  -n, --name    container name                               
  -a, --alias   new container alias                          

Flags
      --version  Show version number                                   
  -h, --help     Show help                                            
  -v, --verbose

Examples:
  lxce alias set -d google                Set alias alice to container 
  -n front -a alice                       front within google domain
\end{minted}
\caption{lxce alias set}
\label{listings: lxce alias set}
\end{listing}
\TODO{Change all descriptions to match [] or <>}

\newpage
\textbf{lxce alias unset}
\begin{listing}[H]
\begin{minted}[bgcolor=background]{text}
Usage: lxce alias unset [options] <flags>

Options
  -d, --domain  container domain                             
  -n, --name    container name                               
  -a, --alias   new container alias                          

Flags
      --version  Show version number                        
  -h, --help     Show help                                  
  -v, --verbose

Examples:
  lxce alias unset -d google -n front  Unset alias to container front 
                                       within google domain
  lxce alias unset -d google -a alice  Unset alias to container with 
                                       alice alias within google 
				       domain
\end{minted}
\caption{lxce alias unset}
\label{listings: lxce alias unset}
\end{listing}

\textbf{lxce alias check}
\begin{listing}[H]
\begin{minted}[bgcolor=background]{text}
Usage: lxce alias check [options] <flags>

Options
  -d, --domain  container domain                             
  -a, --alias   new container alias                          
  -f, --format  output format            ["plain", "json", "csv"]

Flags
      --version  Show version number                                   
  -h, --help     Show help                                             
  -v, --verbose

Examples:
  lxce alias check -d google -a alice  check alice alias existence 
                                       within google domain
\end{minted}
\caption{lxce alias check}
\label{listings: lxce alias check}
\end{listing}


\textbf{lxce delete}
\begin{listing}[H]
\begin{minted}[bgcolor=background]{text}
Usage: lxce delete <options> <flags>

Options
  -g, --global  apply to all containers                         
  -d, --domain  domain name for a group of containers            
  -n, --name    container name                                   
  -a, --alias   container alias                                 
  -y, --yes     yes to questions                                

Flags
      --version  Show version number                            
  -h, --help     Show help                                      
  -v, --verbose

Examples:
  lxce delete --global                   Deletes all containers and
                                         configurations related
  lxce delete -d google                  Deletes all containers within 
                                         google domain
  lxce delete -d google -n still-yellow  Deletes container referenced 
                                         by name
  lxce delete -d google -a alice         Deletes container referenced
                                         by alias
\end{minted}
\caption{lxce delete}
\label{listings: lxce delete}
\end{listing}

\textbf{lxce init}
\begin{listing}[H]
\begin{minted}[bgcolor=background]{text}
Usage: lxce init <flags>

Flags
      --version  Show version number                            
  -h, --help     Show help                                      
  -v, --verbose
\end{minted}
\caption{lxce init}
\label{listings: lxce init}
\end{listing}

\newpage
\textbf{lxce launch}
\begin{listing}[H]
\begin{minted}[bgcolor=background]{text}
Usage: lxce launch <options> <flags>

Options
  -d, --domain   domain for the container/containers         
  -r, --range    range of container (ex: -r 5)             
  -n, --names    names/name of the containers/container                  
  -a, --aliases  aliases/alias of the containers/container               

Flags
      --version  Show version number                                   
  -h, --help     Show help                                             
  -v, --verbose

Examples:
  lxce launch -d google                     Launch one container within 
                                            google with a random name
  lxce launch -d google -r 3                Launch three containers 
                                            within google with 
					    random names
  lxce launch -d google -r 3 -n back front  Launch three containers 
  base                                      within google with 
                                            specified names
  lxce launch -d google -r 3 -n back front  Launch three containers with 
  base -a alice bob eve                     name and alias
                                            specified
  lxce launch -d google -r 3 -a alice bob   Launch three containers 
  eve                                       with random names and 
                                            alias specified
\end{minted}
\caption{lxce launch}
\label{listings: lxce launch}
\end{listing}

\newpage
\textbf{lxce list}
\begin{listing}[H]
\begin{minted}[bgcolor=background]{text}
Usage: lxce <options> <flags>

Format options
==============
-n: "name"
-a: "alias"
-u: "user"
-b: "base"
-r: "ram (MB)"
-p: "ports"
-4: "ipv4"
-6: "ipv6"
-s: "status"
-d: "domain"
-c: "cpu usage (s)"

Options
  -c, --columns  Values to show                                         
  -f, --format   Output format                                          

Flags
      --version  Show version number                                   
  -h, --help     Show help                                             
  -v, --verbose

Examples:
  lxce list -c naubr
  lxce list -f json
\end{minted}
\caption{lxce list}
\label{listings: lxce list}
\end{listing}

\newpage
\textbf{lxce pass}
\begin{listing}[H]
\begin{minted}[bgcolor=background]{text}
Usage: lxce pass <options> <flags>

Options
  -g, --global  Apply to all containers                               
  -d, --domain  Domain name for a group of containers                 
  -n, --name    Container name                                        
  -a, --alias   Container alias                                       
  -p, --plain   plain output                                          

Flags
      --version  Show version number                                  
  -h, --help     Show help                                            
  -v, --verbose

Examples:
  lxce pass --global            Compute all container passwords
  lxce pass --domain google     Compute all domain passwords
  lxce pass -d google -n front  Compute container name password
  lxce pass -d google -a alice  Compute container alias password

\end{minted}
\caption{lxce pass}
\label{listings: lxce pass}
\end{listing}

\newpage
\textbf{lxce proxy}
\begin{listing}[H]
\begin{minted}[bgcolor=background]{text}
Usage: lxce proxy <options> <flags>

Options
  -g, --global  Apply to all containers                                
  -d, --domain  Domain name for a group of containers                  
  -n, --name    Container name                                         
  -a, --alias   Container alias                                        

Flags
      --version  Show version number                                   
  -h, --help     Show help                                             
  -v, --verbose

Examples:
  lxce proxy --global            Restart all containers proxies based 
                                 on their configuration files
  lxce proxy -d google           Restart all domain containers proxies 
                                 based on their configuration files
  lxce proxy -d google -n front  Restart container proxies
  lxce proxy -d google -a alice  Restart container proxies
\end{minted}
\caption{lxce proxy}
\label{listings: lxce proxy}
\end{listing}

\newpage
\textbf{lxce rebase}
\begin{listing}[H]
\begin{minted}[bgcolor=background]{text}
Usage: lxce rebase <options> <flags>

Options
  -g, --global  Applied to all containers                              
  -d, --domain  Domain name for a group of containers                  
  -n, --name    Container name                                         
  -a, --alias   Container alias                                        
  -b, --base    Container base                               

Flags
      --version  Show version number                                   
  -h, --help     Show help                                             
  -v, --verbose

Examples:
  lxce rebase --global                   Applies new base to existing 
                                         containers and future ones
  lxce rebase -d google                  Applies new base to all
                                         containers withing 
                                         google domain
  lxce rebase -d google -n still-yellow  Applies new base to container 
  lxce rebase -d google -a alice         Applies new base to container 
\end{minted}
\caption{lxce rebase}
\label{listings: lxce rebase}
\end{listing}

\newpage
\textbf{lxce show}
\begin{listing}[H]
\begin{minted}[bgcolor=background]{text}
Usage: lxce show <options> <flags>

Options
  -g, --global  Apply to all containers                                
  -d, --domain  Domain name for a group of containers                  
  -n, --name    Container name                                         
  -a, --alias   Container alias                                        
  -e, --extra   Show extra information                

Flags
      --version  Show version number                                   
  -h, --help     Show help                                             
  -v, --verbose

Examples:
  lxce show --global                   Show all containers configurations
  lxce show -d google                  Show all containers configurations 
                                       within domain
  lxce show -d google -n still-yellow  Show container configurations 
                                       defined by name
  lxce show -d google -a alice         Stop container configuration 
                                       defined by alias
\end{minted}
\caption{lxce show}
\label{listings: lxce show}
\end{listing}

\newpage
\textbf{lxce start}
\begin{listing}[H]
\begin{minted}[bgcolor=background]{text}
Usage: lxce start <options> <flags>

Options
  -g, --global  Apply to all containers                                
  -d, --domain  Domain name for a group of containers                  
  -n, --name    Container name                                         
  -a, --alias   Container alias                                        

Flags
      --version  Show version number                                   
  -h, --help     Show help                                             
  -v, --verbose

Examples:
  lxce start --global                   Start all containers
  lxce start -d google                  Start all container within domain
  lxce start -d google -n still-yellow  Start container defined by name
  lxce start -d google -a alice         Start container defined by alias
\end{minted}
\caption{lxce start}
\label{listings: lxce start}
\end{listing}

\newpage
\textbf{lxce stop}
\begin{listing}[H]
\begin{minted}[bgcolor=background]{text}
Usage: lxce stop <options> <flags>

Options
  -g, --global  Apply to all containers                                
  -d, --domain  Domain name for a group of containers                  
  -n, --name    Container name                                         
  -a, --alias   Container alias                                        

Flags
      --version  Show version number                                   
  -h, --help     Show help                                             
  -v, --verbose

Examples:
  lxce stop --global                   Stop all containers
  lxce stop -d google                  Stop all container within domain
  lxce stop -d google -n still-yellow  Stop container defined by name
  lxce stop -d google -a alice         Stop container defined by alias
\end{minted}
\caption{lxce stop}
\label{listings: lxce stop}
\end{listing}

\textbf{lxce uninstall}
\begin{listing}[H]
\begin{minted}[bgcolor=background]{text}
lxce uninstall <options> <flags>

Options
  -y, --yes                                                            

Flags
      --version  Show version number                                   
  -h, --help     Show help                                             
  -v, --verbose
\end{minted}
\caption{lxce uninstall}
\label{listings: lxce uninstall}
\end{listing}
%% END OF COMMANDS %%%%%
%%%%%%%%%%%%%%%%%%%%%%%%
